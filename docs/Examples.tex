\documentclass[11pt]{article}

    \usepackage[breakable]{tcolorbox}
    \usepackage{parskip} % Stop auto-indenting (to mimic markdown behaviour)
    
    \usepackage{iftex}
    \ifPDFTeX
    	\usepackage[T1]{fontenc}
    	\usepackage{mathpazo}
    \else
    	\usepackage{fontspec}
    \fi

    % Basic figure setup, for now with no caption control since it's done
    % automatically by Pandoc (which extracts ![](path) syntax from Markdown).
    \usepackage{graphicx}
    % Maintain compatibility with old templates. Remove in nbconvert 6.0
    \let\Oldincludegraphics\includegraphics
    % Ensure that by default, figures have no caption (until we provide a
    % proper Figure object with a Caption API and a way to capture that
    % in the conversion process - todo).
    \usepackage{caption}
    \DeclareCaptionFormat{nocaption}{}
    \captionsetup{format=nocaption,aboveskip=0pt,belowskip=0pt}

    \usepackage[Export]{adjustbox} % Used to constrain images to a maximum size
    \adjustboxset{max size={0.9\linewidth}{0.9\paperheight}}
    \usepackage{float}
    \floatplacement{figure}{H} % forces figures to be placed at the correct location
    \usepackage{xcolor} % Allow colors to be defined
    \usepackage{enumerate} % Needed for markdown enumerations to work
    \usepackage{geometry} % Used to adjust the document margins
    \usepackage{amsmath} % Equations
    \usepackage{amssymb} % Equations
    \usepackage{textcomp} % defines textquotesingle
    % Hack from http://tex.stackexchange.com/a/47451/13684:
    \AtBeginDocument{%
        \def\PYZsq{\textquotesingle}% Upright quotes in Pygmentized code
    }
    \usepackage{upquote} % Upright quotes for verbatim code
    \usepackage{eurosym} % defines \euro
    \usepackage[mathletters]{ucs} % Extended unicode (utf-8) support
    \usepackage{fancyvrb} % verbatim replacement that allows latex
    \usepackage{grffile} % extends the file name processing of package graphics 
                         % to support a larger range
    \makeatletter % fix for grffile with XeLaTeX
    \def\Gread@@xetex#1{%
      \IfFileExists{"\Gin@base".bb}%
      {\Gread@eps{\Gin@base.bb}}%
      {\Gread@@xetex@aux#1}%
    }
    \makeatother

    % The hyperref package gives us a pdf with properly built
    % internal navigation ('pdf bookmarks' for the table of contents,
    % internal cross-reference links, web links for URLs, etc.)
    \usepackage{hyperref}
    % The default LaTeX title has an obnoxious amount of whitespace. By default,
    % titling removes some of it. It also provides customization options.
    \usepackage{titling}
    \usepackage{longtable} % longtable support required by pandoc >1.10
    \usepackage{booktabs}  % table support for pandoc > 1.12.2
    \usepackage[inline]{enumitem} % IRkernel/repr support (it uses the enumerate* environment)
    \usepackage[normalem]{ulem} % ulem is needed to support strikethroughs (\sout)
                                % normalem makes italics be italics, not underlines
    \usepackage{mathrsfs}
    

    
    % Colors for the hyperref package
    \definecolor{urlcolor}{rgb}{0,.145,.698}
    \definecolor{linkcolor}{rgb}{.71,0.21,0.01}
    \definecolor{citecolor}{rgb}{.12,.54,.11}

    % ANSI colors
    \definecolor{ansi-black}{HTML}{3E424D}
    \definecolor{ansi-black-intense}{HTML}{282C36}
    \definecolor{ansi-red}{HTML}{E75C58}
    \definecolor{ansi-red-intense}{HTML}{B22B31}
    \definecolor{ansi-green}{HTML}{00A250}
    \definecolor{ansi-green-intense}{HTML}{007427}
    \definecolor{ansi-yellow}{HTML}{DDB62B}
    \definecolor{ansi-yellow-intense}{HTML}{B27D12}
    \definecolor{ansi-blue}{HTML}{208FFB}
    \definecolor{ansi-blue-intense}{HTML}{0065CA}
    \definecolor{ansi-magenta}{HTML}{D160C4}
    \definecolor{ansi-magenta-intense}{HTML}{A03196}
    \definecolor{ansi-cyan}{HTML}{60C6C8}
    \definecolor{ansi-cyan-intense}{HTML}{258F8F}
    \definecolor{ansi-white}{HTML}{C5C1B4}
    \definecolor{ansi-white-intense}{HTML}{A1A6B2}
    \definecolor{ansi-default-inverse-fg}{HTML}{FFFFFF}
    \definecolor{ansi-default-inverse-bg}{HTML}{000000}

    % commands and environments needed by pandoc snippets
    % extracted from the output of `pandoc -s`
    \providecommand{\tightlist}{%
      \setlength{\itemsep}{0pt}\setlength{\parskip}{0pt}}
    \DefineVerbatimEnvironment{Highlighting}{Verbatim}{commandchars=\\\{\}}
    % Add ',fontsize=\small' for more characters per line
    \newenvironment{Shaded}{}{}
    \newcommand{\KeywordTok}[1]{\textcolor[rgb]{0.00,0.44,0.13}{\textbf{{#1}}}}
    \newcommand{\DataTypeTok}[1]{\textcolor[rgb]{0.56,0.13,0.00}{{#1}}}
    \newcommand{\DecValTok}[1]{\textcolor[rgb]{0.25,0.63,0.44}{{#1}}}
    \newcommand{\BaseNTok}[1]{\textcolor[rgb]{0.25,0.63,0.44}{{#1}}}
    \newcommand{\FloatTok}[1]{\textcolor[rgb]{0.25,0.63,0.44}{{#1}}}
    \newcommand{\CharTok}[1]{\textcolor[rgb]{0.25,0.44,0.63}{{#1}}}
    \newcommand{\StringTok}[1]{\textcolor[rgb]{0.25,0.44,0.63}{{#1}}}
    \newcommand{\CommentTok}[1]{\textcolor[rgb]{0.38,0.63,0.69}{\textit{{#1}}}}
    \newcommand{\OtherTok}[1]{\textcolor[rgb]{0.00,0.44,0.13}{{#1}}}
    \newcommand{\AlertTok}[1]{\textcolor[rgb]{1.00,0.00,0.00}{\textbf{{#1}}}}
    \newcommand{\FunctionTok}[1]{\textcolor[rgb]{0.02,0.16,0.49}{{#1}}}
    \newcommand{\RegionMarkerTok}[1]{{#1}}
    \newcommand{\ErrorTok}[1]{\textcolor[rgb]{1.00,0.00,0.00}{\textbf{{#1}}}}
    \newcommand{\NormalTok}[1]{{#1}}
    
    % Additional commands for more recent versions of Pandoc
    \newcommand{\ConstantTok}[1]{\textcolor[rgb]{0.53,0.00,0.00}{{#1}}}
    \newcommand{\SpecialCharTok}[1]{\textcolor[rgb]{0.25,0.44,0.63}{{#1}}}
    \newcommand{\VerbatimStringTok}[1]{\textcolor[rgb]{0.25,0.44,0.63}{{#1}}}
    \newcommand{\SpecialStringTok}[1]{\textcolor[rgb]{0.73,0.40,0.53}{{#1}}}
    \newcommand{\ImportTok}[1]{{#1}}
    \newcommand{\DocumentationTok}[1]{\textcolor[rgb]{0.73,0.13,0.13}{\textit{{#1}}}}
    \newcommand{\AnnotationTok}[1]{\textcolor[rgb]{0.38,0.63,0.69}{\textbf{\textit{{#1}}}}}
    \newcommand{\CommentVarTok}[1]{\textcolor[rgb]{0.38,0.63,0.69}{\textbf{\textit{{#1}}}}}
    \newcommand{\VariableTok}[1]{\textcolor[rgb]{0.10,0.09,0.49}{{#1}}}
    \newcommand{\ControlFlowTok}[1]{\textcolor[rgb]{0.00,0.44,0.13}{\textbf{{#1}}}}
    \newcommand{\OperatorTok}[1]{\textcolor[rgb]{0.40,0.40,0.40}{{#1}}}
    \newcommand{\BuiltInTok}[1]{{#1}}
    \newcommand{\ExtensionTok}[1]{{#1}}
    \newcommand{\PreprocessorTok}[1]{\textcolor[rgb]{0.74,0.48,0.00}{{#1}}}
    \newcommand{\AttributeTok}[1]{\textcolor[rgb]{0.49,0.56,0.16}{{#1}}}
    \newcommand{\InformationTok}[1]{\textcolor[rgb]{0.38,0.63,0.69}{\textbf{\textit{{#1}}}}}
    \newcommand{\WarningTok}[1]{\textcolor[rgb]{0.38,0.63,0.69}{\textbf{\textit{{#1}}}}}
    
    
    % Define a nice break command that doesn't care if a line doesn't already
    % exist.
    \def\br{\hspace*{\fill} \\* }
    % Math Jax compatibility definitions
    \def\gt{>}
    \def\lt{<}
    \let\Oldtex\TeX
    \let\Oldlatex\LaTeX
    \renewcommand{\TeX}{\textrm{\Oldtex}}
    \renewcommand{\LaTeX}{\textrm{\Oldlatex}}
    % Document parameters
    % Document title
    \title{Examples}
    
    
    
    
    
% Pygments definitions
\makeatletter
\def\PY@reset{\let\PY@it=\relax \let\PY@bf=\relax%
    \let\PY@ul=\relax \let\PY@tc=\relax%
    \let\PY@bc=\relax \let\PY@ff=\relax}
\def\PY@tok#1{\csname PY@tok@#1\endcsname}
\def\PY@toks#1+{\ifx\relax#1\empty\else%
    \PY@tok{#1}\expandafter\PY@toks\fi}
\def\PY@do#1{\PY@bc{\PY@tc{\PY@ul{%
    \PY@it{\PY@bf{\PY@ff{#1}}}}}}}
\def\PY#1#2{\PY@reset\PY@toks#1+\relax+\PY@do{#2}}

\expandafter\def\csname PY@tok@w\endcsname{\def\PY@tc##1{\textcolor[rgb]{0.73,0.73,0.73}{##1}}}
\expandafter\def\csname PY@tok@c\endcsname{\let\PY@it=\textit\def\PY@tc##1{\textcolor[rgb]{0.25,0.50,0.50}{##1}}}
\expandafter\def\csname PY@tok@cp\endcsname{\def\PY@tc##1{\textcolor[rgb]{0.74,0.48,0.00}{##1}}}
\expandafter\def\csname PY@tok@k\endcsname{\let\PY@bf=\textbf\def\PY@tc##1{\textcolor[rgb]{0.00,0.50,0.00}{##1}}}
\expandafter\def\csname PY@tok@kp\endcsname{\def\PY@tc##1{\textcolor[rgb]{0.00,0.50,0.00}{##1}}}
\expandafter\def\csname PY@tok@kt\endcsname{\def\PY@tc##1{\textcolor[rgb]{0.69,0.00,0.25}{##1}}}
\expandafter\def\csname PY@tok@o\endcsname{\def\PY@tc##1{\textcolor[rgb]{0.40,0.40,0.40}{##1}}}
\expandafter\def\csname PY@tok@ow\endcsname{\let\PY@bf=\textbf\def\PY@tc##1{\textcolor[rgb]{0.67,0.13,1.00}{##1}}}
\expandafter\def\csname PY@tok@nb\endcsname{\def\PY@tc##1{\textcolor[rgb]{0.00,0.50,0.00}{##1}}}
\expandafter\def\csname PY@tok@nf\endcsname{\def\PY@tc##1{\textcolor[rgb]{0.00,0.00,1.00}{##1}}}
\expandafter\def\csname PY@tok@nc\endcsname{\let\PY@bf=\textbf\def\PY@tc##1{\textcolor[rgb]{0.00,0.00,1.00}{##1}}}
\expandafter\def\csname PY@tok@nn\endcsname{\let\PY@bf=\textbf\def\PY@tc##1{\textcolor[rgb]{0.00,0.00,1.00}{##1}}}
\expandafter\def\csname PY@tok@ne\endcsname{\let\PY@bf=\textbf\def\PY@tc##1{\textcolor[rgb]{0.82,0.25,0.23}{##1}}}
\expandafter\def\csname PY@tok@nv\endcsname{\def\PY@tc##1{\textcolor[rgb]{0.10,0.09,0.49}{##1}}}
\expandafter\def\csname PY@tok@no\endcsname{\def\PY@tc##1{\textcolor[rgb]{0.53,0.00,0.00}{##1}}}
\expandafter\def\csname PY@tok@nl\endcsname{\def\PY@tc##1{\textcolor[rgb]{0.63,0.63,0.00}{##1}}}
\expandafter\def\csname PY@tok@ni\endcsname{\let\PY@bf=\textbf\def\PY@tc##1{\textcolor[rgb]{0.60,0.60,0.60}{##1}}}
\expandafter\def\csname PY@tok@na\endcsname{\def\PY@tc##1{\textcolor[rgb]{0.49,0.56,0.16}{##1}}}
\expandafter\def\csname PY@tok@nt\endcsname{\let\PY@bf=\textbf\def\PY@tc##1{\textcolor[rgb]{0.00,0.50,0.00}{##1}}}
\expandafter\def\csname PY@tok@nd\endcsname{\def\PY@tc##1{\textcolor[rgb]{0.67,0.13,1.00}{##1}}}
\expandafter\def\csname PY@tok@s\endcsname{\def\PY@tc##1{\textcolor[rgb]{0.73,0.13,0.13}{##1}}}
\expandafter\def\csname PY@tok@sd\endcsname{\let\PY@it=\textit\def\PY@tc##1{\textcolor[rgb]{0.73,0.13,0.13}{##1}}}
\expandafter\def\csname PY@tok@si\endcsname{\let\PY@bf=\textbf\def\PY@tc##1{\textcolor[rgb]{0.73,0.40,0.53}{##1}}}
\expandafter\def\csname PY@tok@se\endcsname{\let\PY@bf=\textbf\def\PY@tc##1{\textcolor[rgb]{0.73,0.40,0.13}{##1}}}
\expandafter\def\csname PY@tok@sr\endcsname{\def\PY@tc##1{\textcolor[rgb]{0.73,0.40,0.53}{##1}}}
\expandafter\def\csname PY@tok@ss\endcsname{\def\PY@tc##1{\textcolor[rgb]{0.10,0.09,0.49}{##1}}}
\expandafter\def\csname PY@tok@sx\endcsname{\def\PY@tc##1{\textcolor[rgb]{0.00,0.50,0.00}{##1}}}
\expandafter\def\csname PY@tok@m\endcsname{\def\PY@tc##1{\textcolor[rgb]{0.40,0.40,0.40}{##1}}}
\expandafter\def\csname PY@tok@gh\endcsname{\let\PY@bf=\textbf\def\PY@tc##1{\textcolor[rgb]{0.00,0.00,0.50}{##1}}}
\expandafter\def\csname PY@tok@gu\endcsname{\let\PY@bf=\textbf\def\PY@tc##1{\textcolor[rgb]{0.50,0.00,0.50}{##1}}}
\expandafter\def\csname PY@tok@gd\endcsname{\def\PY@tc##1{\textcolor[rgb]{0.63,0.00,0.00}{##1}}}
\expandafter\def\csname PY@tok@gi\endcsname{\def\PY@tc##1{\textcolor[rgb]{0.00,0.63,0.00}{##1}}}
\expandafter\def\csname PY@tok@gr\endcsname{\def\PY@tc##1{\textcolor[rgb]{1.00,0.00,0.00}{##1}}}
\expandafter\def\csname PY@tok@ge\endcsname{\let\PY@it=\textit}
\expandafter\def\csname PY@tok@gs\endcsname{\let\PY@bf=\textbf}
\expandafter\def\csname PY@tok@gp\endcsname{\let\PY@bf=\textbf\def\PY@tc##1{\textcolor[rgb]{0.00,0.00,0.50}{##1}}}
\expandafter\def\csname PY@tok@go\endcsname{\def\PY@tc##1{\textcolor[rgb]{0.53,0.53,0.53}{##1}}}
\expandafter\def\csname PY@tok@gt\endcsname{\def\PY@tc##1{\textcolor[rgb]{0.00,0.27,0.87}{##1}}}
\expandafter\def\csname PY@tok@err\endcsname{\def\PY@bc##1{\setlength{\fboxsep}{0pt}\fcolorbox[rgb]{1.00,0.00,0.00}{1,1,1}{\strut ##1}}}
\expandafter\def\csname PY@tok@kc\endcsname{\let\PY@bf=\textbf\def\PY@tc##1{\textcolor[rgb]{0.00,0.50,0.00}{##1}}}
\expandafter\def\csname PY@tok@kd\endcsname{\let\PY@bf=\textbf\def\PY@tc##1{\textcolor[rgb]{0.00,0.50,0.00}{##1}}}
\expandafter\def\csname PY@tok@kn\endcsname{\let\PY@bf=\textbf\def\PY@tc##1{\textcolor[rgb]{0.00,0.50,0.00}{##1}}}
\expandafter\def\csname PY@tok@kr\endcsname{\let\PY@bf=\textbf\def\PY@tc##1{\textcolor[rgb]{0.00,0.50,0.00}{##1}}}
\expandafter\def\csname PY@tok@bp\endcsname{\def\PY@tc##1{\textcolor[rgb]{0.00,0.50,0.00}{##1}}}
\expandafter\def\csname PY@tok@fm\endcsname{\def\PY@tc##1{\textcolor[rgb]{0.00,0.00,1.00}{##1}}}
\expandafter\def\csname PY@tok@vc\endcsname{\def\PY@tc##1{\textcolor[rgb]{0.10,0.09,0.49}{##1}}}
\expandafter\def\csname PY@tok@vg\endcsname{\def\PY@tc##1{\textcolor[rgb]{0.10,0.09,0.49}{##1}}}
\expandafter\def\csname PY@tok@vi\endcsname{\def\PY@tc##1{\textcolor[rgb]{0.10,0.09,0.49}{##1}}}
\expandafter\def\csname PY@tok@vm\endcsname{\def\PY@tc##1{\textcolor[rgb]{0.10,0.09,0.49}{##1}}}
\expandafter\def\csname PY@tok@sa\endcsname{\def\PY@tc##1{\textcolor[rgb]{0.73,0.13,0.13}{##1}}}
\expandafter\def\csname PY@tok@sb\endcsname{\def\PY@tc##1{\textcolor[rgb]{0.73,0.13,0.13}{##1}}}
\expandafter\def\csname PY@tok@sc\endcsname{\def\PY@tc##1{\textcolor[rgb]{0.73,0.13,0.13}{##1}}}
\expandafter\def\csname PY@tok@dl\endcsname{\def\PY@tc##1{\textcolor[rgb]{0.73,0.13,0.13}{##1}}}
\expandafter\def\csname PY@tok@s2\endcsname{\def\PY@tc##1{\textcolor[rgb]{0.73,0.13,0.13}{##1}}}
\expandafter\def\csname PY@tok@sh\endcsname{\def\PY@tc##1{\textcolor[rgb]{0.73,0.13,0.13}{##1}}}
\expandafter\def\csname PY@tok@s1\endcsname{\def\PY@tc##1{\textcolor[rgb]{0.73,0.13,0.13}{##1}}}
\expandafter\def\csname PY@tok@mb\endcsname{\def\PY@tc##1{\textcolor[rgb]{0.40,0.40,0.40}{##1}}}
\expandafter\def\csname PY@tok@mf\endcsname{\def\PY@tc##1{\textcolor[rgb]{0.40,0.40,0.40}{##1}}}
\expandafter\def\csname PY@tok@mh\endcsname{\def\PY@tc##1{\textcolor[rgb]{0.40,0.40,0.40}{##1}}}
\expandafter\def\csname PY@tok@mi\endcsname{\def\PY@tc##1{\textcolor[rgb]{0.40,0.40,0.40}{##1}}}
\expandafter\def\csname PY@tok@il\endcsname{\def\PY@tc##1{\textcolor[rgb]{0.40,0.40,0.40}{##1}}}
\expandafter\def\csname PY@tok@mo\endcsname{\def\PY@tc##1{\textcolor[rgb]{0.40,0.40,0.40}{##1}}}
\expandafter\def\csname PY@tok@ch\endcsname{\let\PY@it=\textit\def\PY@tc##1{\textcolor[rgb]{0.25,0.50,0.50}{##1}}}
\expandafter\def\csname PY@tok@cm\endcsname{\let\PY@it=\textit\def\PY@tc##1{\textcolor[rgb]{0.25,0.50,0.50}{##1}}}
\expandafter\def\csname PY@tok@cpf\endcsname{\let\PY@it=\textit\def\PY@tc##1{\textcolor[rgb]{0.25,0.50,0.50}{##1}}}
\expandafter\def\csname PY@tok@c1\endcsname{\let\PY@it=\textit\def\PY@tc##1{\textcolor[rgb]{0.25,0.50,0.50}{##1}}}
\expandafter\def\csname PY@tok@cs\endcsname{\let\PY@it=\textit\def\PY@tc##1{\textcolor[rgb]{0.25,0.50,0.50}{##1}}}

\def\PYZbs{\char`\\}
\def\PYZus{\char`\_}
\def\PYZob{\char`\{}
\def\PYZcb{\char`\}}
\def\PYZca{\char`\^}
\def\PYZam{\char`\&}
\def\PYZlt{\char`\<}
\def\PYZgt{\char`\>}
\def\PYZsh{\char`\#}
\def\PYZpc{\char`\%}
\def\PYZdl{\char`\$}
\def\PYZhy{\char`\-}
\def\PYZsq{\char`\'}
\def\PYZdq{\char`\"}
\def\PYZti{\char`\~}
% for compatibility with earlier versions
\def\PYZat{@}
\def\PYZlb{[}
\def\PYZrb{]}
\makeatother


    % For linebreaks inside Verbatim environment from package fancyvrb. 
    \makeatletter
        \newbox\Wrappedcontinuationbox 
        \newbox\Wrappedvisiblespacebox 
        \newcommand*\Wrappedvisiblespace {\textcolor{red}{\textvisiblespace}} 
        \newcommand*\Wrappedcontinuationsymbol {\textcolor{red}{\llap{\tiny$\m@th\hookrightarrow$}}} 
        \newcommand*\Wrappedcontinuationindent {3ex } 
        \newcommand*\Wrappedafterbreak {\kern\Wrappedcontinuationindent\copy\Wrappedcontinuationbox} 
        % Take advantage of the already applied Pygments mark-up to insert 
        % potential linebreaks for TeX processing. 
        %        {, <, #, %, $, ' and ": go to next line. 
        %        _, }, ^, &, >, - and ~: stay at end of broken line. 
        % Use of \textquotesingle for straight quote. 
        \newcommand*\Wrappedbreaksatspecials {% 
            \def\PYGZus{\discretionary{\char`\_}{\Wrappedafterbreak}{\char`\_}}% 
            \def\PYGZob{\discretionary{}{\Wrappedafterbreak\char`\{}{\char`\{}}% 
            \def\PYGZcb{\discretionary{\char`\}}{\Wrappedafterbreak}{\char`\}}}% 
            \def\PYGZca{\discretionary{\char`\^}{\Wrappedafterbreak}{\char`\^}}% 
            \def\PYGZam{\discretionary{\char`\&}{\Wrappedafterbreak}{\char`\&}}% 
            \def\PYGZlt{\discretionary{}{\Wrappedafterbreak\char`\<}{\char`\<}}% 
            \def\PYGZgt{\discretionary{\char`\>}{\Wrappedafterbreak}{\char`\>}}% 
            \def\PYGZsh{\discretionary{}{\Wrappedafterbreak\char`\#}{\char`\#}}% 
            \def\PYGZpc{\discretionary{}{\Wrappedafterbreak\char`\%}{\char`\%}}% 
            \def\PYGZdl{\discretionary{}{\Wrappedafterbreak\char`\$}{\char`\$}}% 
            \def\PYGZhy{\discretionary{\char`\-}{\Wrappedafterbreak}{\char`\-}}% 
            \def\PYGZsq{\discretionary{}{\Wrappedafterbreak\textquotesingle}{\textquotesingle}}% 
            \def\PYGZdq{\discretionary{}{\Wrappedafterbreak\char`\"}{\char`\"}}% 
            \def\PYGZti{\discretionary{\char`\~}{\Wrappedafterbreak}{\char`\~}}% 
        } 
        % Some characters . , ; ? ! / are not pygmentized. 
        % This macro makes them "active" and they will insert potential linebreaks 
        \newcommand*\Wrappedbreaksatpunct {% 
            \lccode`\~`\.\lowercase{\def~}{\discretionary{\hbox{\char`\.}}{\Wrappedafterbreak}{\hbox{\char`\.}}}% 
            \lccode`\~`\,\lowercase{\def~}{\discretionary{\hbox{\char`\,}}{\Wrappedafterbreak}{\hbox{\char`\,}}}% 
            \lccode`\~`\;\lowercase{\def~}{\discretionary{\hbox{\char`\;}}{\Wrappedafterbreak}{\hbox{\char`\;}}}% 
            \lccode`\~`\:\lowercase{\def~}{\discretionary{\hbox{\char`\:}}{\Wrappedafterbreak}{\hbox{\char`\:}}}% 
            \lccode`\~`\?\lowercase{\def~}{\discretionary{\hbox{\char`\?}}{\Wrappedafterbreak}{\hbox{\char`\?}}}% 
            \lccode`\~`\!\lowercase{\def~}{\discretionary{\hbox{\char`\!}}{\Wrappedafterbreak}{\hbox{\char`\!}}}% 
            \lccode`\~`\/\lowercase{\def~}{\discretionary{\hbox{\char`\/}}{\Wrappedafterbreak}{\hbox{\char`\/}}}% 
            \catcode`\.\active
            \catcode`\,\active 
            \catcode`\;\active
            \catcode`\:\active
            \catcode`\?\active
            \catcode`\!\active
            \catcode`\/\active 
            \lccode`\~`\~ 	
        }
    \makeatother

    \let\OriginalVerbatim=\Verbatim
    \makeatletter
    \renewcommand{\Verbatim}[1][1]{%
        %\parskip\z@skip
        \sbox\Wrappedcontinuationbox {\Wrappedcontinuationsymbol}%
        \sbox\Wrappedvisiblespacebox {\FV@SetupFont\Wrappedvisiblespace}%
        \def\FancyVerbFormatLine ##1{\hsize\linewidth
            \vtop{\raggedright\hyphenpenalty\z@\exhyphenpenalty\z@
                \doublehyphendemerits\z@\finalhyphendemerits\z@
                \strut ##1\strut}%
        }%
        % If the linebreak is at a space, the latter will be displayed as visible
        % space at end of first line, and a continuation symbol starts next line.
        % Stretch/shrink are however usually zero for typewriter font.
        \def\FV@Space {%
            \nobreak\hskip\z@ plus\fontdimen3\font minus\fontdimen4\font
            \discretionary{\copy\Wrappedvisiblespacebox}{\Wrappedafterbreak}
            {\kern\fontdimen2\font}%
        }%
        
        % Allow breaks at special characters using \PYG... macros.
        \Wrappedbreaksatspecials
        % Breaks at punctuation characters . , ; ? ! and / need catcode=\active 	
        \OriginalVerbatim[#1,codes*=\Wrappedbreaksatpunct]%
    }
    \makeatother

    % Exact colors from NB
    \definecolor{incolor}{HTML}{303F9F}
    \definecolor{outcolor}{HTML}{D84315}
    \definecolor{cellborder}{HTML}{CFCFCF}
    \definecolor{cellbackground}{HTML}{F7F7F7}
    
    % prompt
    \makeatletter
    \newcommand{\boxspacing}{\kern\kvtcb@left@rule\kern\kvtcb@boxsep}
    \makeatother
    \newcommand{\prompt}[4]{
        \ttfamily\llap{{\color{#2}[#3]:\hspace{3pt}#4}}\vspace{-\baselineskip}
    }
    

    
    % Prevent overflowing lines due to hard-to-break entities
    \sloppy 
    % Setup hyperref package
    \hypersetup{
      breaklinks=true,  % so long urls are correctly broken across lines
      colorlinks=true,
      urlcolor=urlcolor,
      linkcolor=linkcolor,
      citecolor=citecolor,
      }
    % Slightly bigger margins than the latex defaults
    
    \geometry{verbose,tmargin=1in,bmargin=1in,lmargin=1in,rmargin=1in}
    
    

\begin{document}
    
    \maketitle
    
    

    
    \hypertarget{pytex}{%
\section{Pytex}\label{pytex}}

A light-weight pythonic wrapper around LaTeX specifically to provide
python-like interface to write mathemical latex in Jupyter Notebooks.\\
\textbf{NOTE:} Only Jupyter Notebooks are supported for now. Tests done
in notebooks too

\hypertarget{usage}{%
\subsection{Usage}\label{usage}}

The usage is best presented by examples\\
The next cells show some use cases

It is recommended to use the \texttt{Python-Markdown} extension to
render the latex inside a markdown cell.\\
You can enclose a python statement inside double curly braces (see
Python-markdown on Jupyter website) \texttt{\{\ \{\ statement\ \}\ \}}
inside a markdown cell if this extension is enable \textbf{and} the
notebook is trusted.

As an alternative: the function \texttt{latex} returns a
\texttt{iPython.core.display.Latex} object. It has attribute
\texttt{data} which can be printed in a code cell. This string can be
enclosed in \texttt{\$\$\ \textless{}output\textgreater{}\ \$\$} in a
markdown cell

\texttt{platex} prints the latex that can be directly copy pasted inside
a markdown cell.

    \hypertarget{variables}{%
\subsection{Variables}\label{variables}}

\texttt{Var} class is the basic class of the package which defines
\texttt{+},\texttt{-}, function calling etc. and constructs the LaTeX
AST internally.\\
Calling \texttt{latex} with any \texttt{Var} as argument gives the LaTeX
in a platform dependent object. Since only Jupyter Notebooks are
supported for now, it returns an \texttt{IPython.core.display.Latex}
object

    \begin{tcolorbox}[breakable, size=fbox, boxrule=1pt, pad at break*=1mm,colback=cellbackground, colframe=cellborder]
\prompt{In}{incolor}{1}{\boxspacing}
\begin{Verbatim}[commandchars=\\\{\}]
\PY{k+kn}{from} \PY{n+nn}{pytex} \PY{k+kn}{import} \PY{n}{Var}
\PY{k+kn}{from} \PY{n+nn}{pytex}\PY{n+nn}{.}\PY{n+nn}{platforms}\PY{n+nn}{.}\PY{n+nn}{jupyter} \PY{k+kn}{import} \PY{n}{latex}
\PY{n}{a} \PY{o}{=} \PY{n}{Var}\PY{p}{(}\PY{l+s+s1}{\PYZsq{}}\PY{l+s+s1}{a}\PY{l+s+s1}{\PYZsq{}}\PY{p}{)}
\PY{n+nb}{print}\PY{p}{(}\PY{n}{latex}\PY{p}{(}\PY{n}{a}\PY{p}{)}\PY{o}{.}\PY{n}{data}\PY{p}{)}
\end{Verbatim}
\end{tcolorbox}

    \begin{Verbatim}[commandchars=\\\{\}]
\textbackslash{}begin\{gather\}a\textbackslash{}end\{gather\}
    \end{Verbatim}

    \[
\begin{gather}a\end{gather}
\]

    \texttt{platex} prints the latex which can be put inside a markdown
cell. we will use that

    \begin{tcolorbox}[breakable, size=fbox, boxrule=1pt, pad at break*=1mm,colback=cellbackground, colframe=cellborder]
\prompt{In}{incolor}{2}{\boxspacing}
\begin{Verbatim}[commandchars=\\\{\}]
\PY{k+kn}{from} \PY{n+nn}{pytex}\PY{n+nn}{.}\PY{n+nn}{platforms}\PY{n+nn}{.}\PY{n+nn}{jupyter} \PY{k+kn}{import} \PY{n}{platex}
\PY{n}{platex}\PY{p}{(}\PY{n}{a}\PY{p}{)}
\end{Verbatim}
\end{tcolorbox}

    \begin{Verbatim}[commandchars=\\\{\}]
\textbackslash{}begin\{gather\}a\textbackslash{}end\{gather\}
    \end{Verbatim}

    \[ \begin{gather}a\end{gather} \]

    \texttt{makeVar} is a convenience function to create multiple
\texttt{Var} variables easily.\\
A list of operators supported: + \texttt{+}: Adding + \texttt{-}:
Subtraction + \texttt{*}: Multiplication + \texttt{pow}: Exponents +
\texttt{\^{}}: Exponents (another way) + \texttt{==}: Equality
(\texttt{=} cannot be used as assignment is a core feature of python) +
\texttt{/}: Fractions + \texttt{\textbar{}}: Space in generated latex +
\texttt{()}: Function calling with arbitrary args +
\texttt{\textless{}}, \texttt{\textgreater{}}, \texttt{\textless{}=},
\texttt{\textgreater{}=}: Ordering operators

In some cases you can directly use literal constants (numbers, strings
etc) like
\texttt{\textquotesingle{}a\textquotesingle{},\ \textquotesingle{}bc\textquotesingle{},\ 1,\ 2}.
Use cases where this is safe will be discussed later.\\
It is however usually safer to wrap all your constants inside
\texttt{Var} using \texttt{makeVar} for concise syntax as shown below.
(eg. binding const \texttt{2} to instance \texttt{\_2} of \texttt{Var}
as shown below)\\
\texttt{Var} instances also can be called with arbitrary parameters for
them to render as functions

    \begin{tcolorbox}[breakable, size=fbox, boxrule=1pt, pad at break*=1mm,colback=cellbackground, colframe=cellborder]
\prompt{In}{incolor}{3}{\boxspacing}
\begin{Verbatim}[commandchars=\\\{\}]
\PY{k+kn}{from} \PY{n+nn}{pytex} \PY{k+kn}{import} \PY{n}{makeVar}
\PY{n}{x}\PY{p}{,}\PY{n}{y}\PY{p}{,}\PY{n}{z}\PY{p}{,}\PY{n}{f}\PY{p}{,}\PY{n}{ab}\PY{p}{,}\PY{n}{\PYZus{}2} \PY{o}{=} \PY{n}{makeVar}\PY{p}{(}\PY{l+s+s1}{\PYZsq{}}\PY{l+s+s1}{x}\PY{l+s+s1}{\PYZsq{}}\PY{p}{,}\PY{l+s+s1}{\PYZsq{}}\PY{l+s+s1}{y}\PY{l+s+s1}{\PYZsq{}}\PY{p}{,}\PY{l+s+s1}{\PYZsq{}}\PY{l+s+s1}{z}\PY{l+s+s1}{\PYZsq{}}\PY{p}{,}\PY{l+s+s1}{\PYZsq{}}\PY{l+s+s1}{f}\PY{l+s+s1}{\PYZsq{}}\PY{p}{,}\PY{l+s+s1}{\PYZsq{}}\PY{l+s+s1}{ab}\PY{l+s+s1}{\PYZsq{}}\PY{p}{,} \PY{l+m+mi}{2}\PY{p}{)}
\PY{n}{op} \PY{o}{=} \PY{n}{f}\PY{p}{(}\PY{n}{x} \PY{o}{+} \PY{n}{y}\PY{o}{|}\PY{n}{z} \PY{o}{+} \PY{n}{ab}\PY{p}{)} \PY{o}{==} \PY{n}{x}\PY{o}{*}\PY{o}{*}\PY{l+m+mi}{2} \PY{o}{+} \PY{n}{x} \PY{o}{+} \PY{n}{ab}
\PY{n}{platex}\PY{p}{(}\PY{n}{op}\PY{p}{)}
\end{Verbatim}
\end{tcolorbox}

    \begin{Verbatim}[commandchars=\\\{\}]
\textbackslash{}begin\{gather\}f(x + y\textbackslash{}hspace\{1mm\}z + ab) = x\^{}2 + x + ab\textbackslash{}end\{gather\}
    \end{Verbatim}

    \[ \begin{gather}f(x + y\hspace{1mm}z + ab) = x^2 + x + ab\end{gather} \]

    Ordering Operators Example

    \begin{tcolorbox}[breakable, size=fbox, boxrule=1pt, pad at break*=1mm,colback=cellbackground, colframe=cellborder]
\prompt{In}{incolor}{4}{\boxspacing}
\begin{Verbatim}[commandchars=\\\{\}]
\PY{n}{platex}\PY{p}{(}\PY{n}{x} \PY{o}{\PYZgt{}}\PY{o}{=} \PY{n}{y}\PY{p}{)}
\end{Verbatim}
\end{tcolorbox}

    \begin{Verbatim}[commandchars=\\\{\}]
\textbackslash{}begin\{gather\}x \textbackslash{}geq y\textbackslash{}end\{gather\}
    \end{Verbatim}

    \[ \begin{gather}x \geq y\end{gather} \]

    String Argument passed to \texttt{Var} or \texttt{makeVar} is rendered
in the latex. It is recommended to map arguments to approximately same
names of \texttt{Var} instances

    \begin{tcolorbox}[breakable, size=fbox, boxrule=1pt, pad at break*=1mm,colback=cellbackground, colframe=cellborder]
\prompt{In}{incolor}{5}{\boxspacing}
\begin{Verbatim}[commandchars=\\\{\}]
\PY{n}{a} \PY{o}{=} \PY{n}{Var}\PY{p}{(}\PY{l+s+s1}{\PYZsq{}}\PY{l+s+s1}{a}\PY{l+s+s1}{\PYZsq{}}\PY{p}{)}
\PY{n}{b} \PY{o}{=} \PY{n}{Var}\PY{p}{(}\PY{l+s+s1}{\PYZsq{}}\PY{l+s+s1}{x}\PY{l+s+s1}{\PYZsq{}}\PY{p}{)}
\PY{n}{c} \PY{o}{=} \PY{n}{makeVar}\PY{p}{(}\PY{l+s+s1}{\PYZsq{}}\PY{l+s+s1}{c}\PY{l+s+s1}{\PYZsq{}}\PY{p}{)}

\PY{c+c1}{\PYZsh{} b will render as x in the latex, binding different names to variables is not recommended}
\PY{n}{platex}\PY{p}{(}\PY{n}{a} \PY{o}{==} \PY{n}{b} \PY{o}{+} \PY{n}{c}\PY{p}{)}
\end{Verbatim}
\end{tcolorbox}

    \begin{Verbatim}[commandchars=\\\{\}]
\textbackslash{}begin\{gather\}a = x + c\textbackslash{}end\{gather\}
    \end{Verbatim}

    \[ \begin{gather}a = x + c\end{gather} \]

    \texttt{==} is overloaded version of \texttt{equals} method. Use
whichever you prefer

    \begin{tcolorbox}[breakable, size=fbox, boxrule=1pt, pad at break*=1mm,colback=cellbackground, colframe=cellborder]
\prompt{In}{incolor}{7}{\boxspacing}
\begin{Verbatim}[commandchars=\\\{\}]
\PY{n}{platex}\PY{p}{(}\PY{n}{a}\PY{o}{.}\PY{n}{equals}\PY{p}{(}\PY{n}{b} \PY{o}{+} \PY{n}{c}\PY{p}{)}\PY{p}{)}
\end{Verbatim}
\end{tcolorbox}

    \begin{Verbatim}[commandchars=\\\{\}]
\textbackslash{}begin\{gather\}a = x + c\textbackslash{}end\{gather\}
    \end{Verbatim}

    You can also use Vectors. \texttt{Vector} class is used.
\texttt{makeVector} convenience function similarly supported

    \begin{tcolorbox}[breakable, size=fbox, boxrule=1pt, pad at break*=1mm,colback=cellbackground, colframe=cellborder]
\prompt{In}{incolor}{9}{\boxspacing}
\begin{Verbatim}[commandchars=\\\{\}]
\PY{k+kn}{from} \PY{n+nn}{pytex} \PY{k+kn}{import} \PY{n}{Vector}\PY{p}{,} \PY{n}{makeVector}
\PY{n}{f}\PY{p}{,}\PY{n}{x}\PY{p}{,}\PY{n}{y} \PY{o}{=} \PY{n}{makeVector}\PY{p}{(}\PY{l+s+s1}{\PYZsq{}}\PY{l+s+s1}{f}\PY{l+s+s1}{\PYZsq{}}\PY{p}{,}\PY{l+s+s1}{\PYZsq{}}\PY{l+s+s1}{x}\PY{l+s+s1}{\PYZsq{}}\PY{p}{,}\PY{l+s+s1}{\PYZsq{}}\PY{l+s+s1}{y}\PY{l+s+s1}{\PYZsq{}}\PY{p}{)}
\PY{n}{\PYZus{}1} \PY{o}{=} \PY{n}{makeVar}\PY{p}{(}\PY{l+m+mi}{1}\PY{p}{)}
\PY{n}{op} \PY{o}{=} \PY{n}{f}\PY{p}{(}\PY{n}{x}\PY{p}{,}\PY{n}{y}\PY{p}{)} \PY{o}{==} \PY{n}{x}\PY{o}{*}\PY{o}{*}\PY{n}{y} \PY{o}{+} \PY{n}{x}\PY{o}{*}\PY{n}{y} \PY{o}{+} \PY{n}{\PYZus{}1}
\PY{n}{platex}\PY{p}{(}\PY{n}{op}\PY{p}{)}
\end{Verbatim}
\end{tcolorbox}

    \begin{Verbatim}[commandchars=\\\{\}]
\textbackslash{}begin\{gather\}\textbackslash{}vec\{\textbackslash{}mathbf\{f\}\}(\textbackslash{}vec\{\textbackslash{}mathbf\{x\}\},\textbackslash{}vec\{\textbackslash{}mathbf\{y\}\}) =
\textbackslash{}vec\{\textbackslash{}mathbf\{x\}\}\^{}\textbackslash{}vec\{\textbackslash{}mathbf\{y\}\} + \textbackslash{}vec\{\textbackslash{}mathbf\{x\}\} * \textbackslash{}vec\{\textbackslash{}mathbf\{y\}\} +
1\textbackslash{}end\{gather\}
    \end{Verbatim}

    \[ \begin{gather}\vec{\mathbf{f}}(\vec{\mathbf{x}},\vec{\mathbf{y}}) = \vec{\mathbf{x}}^\vec{\mathbf{y}} + \vec{\mathbf{x}} * \vec{\mathbf{y}} + 1\end{gather} \]

    \hypertarget{series}{%
\subsection{Series}\label{series}}

\hypertarget{summation}{%
\subsubsection{Summation}\label{summation}}

\textbf{NOTE:} upper and lower limits are optional. (Shown in
\texttt{Product})

    \begin{tcolorbox}[breakable, size=fbox, boxrule=1pt, pad at break*=1mm,colback=cellbackground, colframe=cellborder]
\prompt{In}{incolor}{10}{\boxspacing}
\begin{Verbatim}[commandchars=\\\{\}]
\PY{k+kn}{from} \PY{n+nn}{pytex} \PY{k+kn}{import} \PY{n}{Sum}
\PY{n}{i}\PY{p}{,} \PY{n}{\PYZus{}1}\PY{p}{,} \PY{n}{\PYZus{}10} \PY{o}{=} \PY{n}{makeVar}\PY{p}{(}\PY{l+s+s1}{\PYZsq{}}\PY{l+s+s1}{i}\PY{l+s+s1}{\PYZsq{}}\PY{p}{,} \PY{l+m+mi}{1}\PY{p}{,} \PY{l+m+mi}{10}\PY{p}{)}
\PY{n}{op} \PY{o}{=} \PY{n}{Sum}\PY{p}{(}\PY{n}{i}\PY{o}{*}\PY{o}{*}\PY{l+m+mi}{2}\PY{o}{+}\PY{n}{i}\PY{o}{+}\PY{l+m+mi}{1}\PY{p}{,} \PY{n}{i}\PY{p}{,} \PY{n}{\PYZus{}1}\PY{p}{,} \PY{n}{\PYZus{}10}\PY{p}{)}
\PY{n}{platex}\PY{p}{(}\PY{n}{op}\PY{p}{)}
\end{Verbatim}
\end{tcolorbox}

    \begin{Verbatim}[commandchars=\\\{\}]
\textbackslash{}begin\{gather\}\textbackslash{}sum\_\{i=1\}\^{}\{10\} i\^{}2 + i + 1\textbackslash{}end\{gather\}
    \end{Verbatim}

    \[ \begin{gather}\sum_{i=1}^{10} i^2 + i + 1\end{gather} \]

    \hypertarget{product}{%
\subsubsection{Product}\label{product}}

    \begin{tcolorbox}[breakable, size=fbox, boxrule=1pt, pad at break*=1mm,colback=cellbackground, colframe=cellborder]
\prompt{In}{incolor}{11}{\boxspacing}
\begin{Verbatim}[commandchars=\\\{\}]
\PY{k+kn}{from} \PY{n+nn}{pytex} \PY{k+kn}{import} \PY{n}{Product}
\PY{n}{i} \PY{o}{=} \PY{n}{makeVar}\PY{p}{(}\PY{l+s+s1}{\PYZsq{}}\PY{l+s+s1}{i}\PY{l+s+s1}{\PYZsq{}}\PY{p}{)}
\PY{n}{op} \PY{o}{=} \PY{n}{Product}\PY{p}{(}\PY{n+nb}{pow}\PY{p}{(}\PY{n}{i}\PY{p}{,}\PY{l+m+mi}{2}\PY{p}{)}\PY{o}{+}\PY{n}{i}\PY{o}{+}\PY{l+m+mi}{1}\PY{p}{,} \PY{n}{i}\PY{p}{)}
\PY{n}{platex}\PY{p}{(}\PY{n}{op}\PY{p}{)}
\end{Verbatim}
\end{tcolorbox}

    \begin{Verbatim}[commandchars=\\\{\}]
\textbackslash{}begin\{gather\}\textbackslash{}prod\_\{i\} i\^{}2 + i + 1\textbackslash{}end\{gather\}
    \end{Verbatim}

    \[ \begin{gather}\prod_{i} i^2 + i + 1\end{gather} \]

    \hypertarget{derivatives-and-partial-derivatives}{%
\subsection{Derivatives and Partial
Derivatives}\label{derivatives-and-partial-derivatives}}

Similar to \texttt{Var} we have \texttt{Derivative} and \texttt{Partial}
implementations. \texttt{makeDerivative} and \texttt{makePartial} are
parallel implementations of \texttt{makeVar}

    \begin{tcolorbox}[breakable, size=fbox, boxrule=1pt, pad at break*=1mm,colback=cellbackground, colframe=cellborder]
\prompt{In}{incolor}{12}{\boxspacing}
\begin{Verbatim}[commandchars=\\\{\}]
\PY{k+kn}{from} \PY{n+nn}{pytex} \PY{k+kn}{import} \PY{n}{makeDerivative}

\PY{c+c1}{\PYZsh{} NOTE: makeDerivative takes args in forms of tuples of size two}
\PY{c+c1}{\PYZsh{}       the first element in the tuple is the name of the differentiating variable}
\PY{c+c1}{\PYZsh{}       the second element is the degree of the variable}
\PY{c+c1}{\PYZsh{} NOTE: degree of 1 is not shown in latex}

\PY{n}{dx}\PY{p}{,} \PY{n}{dy} \PY{o}{=} \PY{n}{makeDerivative}\PY{p}{(}\PY{p}{(}\PY{l+s+s1}{\PYZsq{}}\PY{l+s+s1}{x}\PY{l+s+s1}{\PYZsq{}}\PY{p}{,} \PY{l+m+mi}{1}\PY{p}{)}\PY{p}{,} \PY{p}{(}\PY{l+s+s1}{\PYZsq{}}\PY{l+s+s1}{y}\PY{l+s+s1}{\PYZsq{}}\PY{p}{,} \PY{l+m+mi}{2}\PY{p}{)}\PY{p}{)}
\PY{n}{x}\PY{p}{,} \PY{n}{y} \PY{o}{=} \PY{n}{makeVar}\PY{p}{(}\PY{l+s+s1}{\PYZsq{}}\PY{l+s+s1}{x}\PY{l+s+s1}{\PYZsq{}}\PY{p}{,} \PY{l+s+s1}{\PYZsq{}}\PY{l+s+s1}{y}\PY{l+s+s1}{\PYZsq{}}\PY{p}{)}
\PY{n}{op} \PY{o}{=} \PY{n}{dx} \PY{o}{|} \PY{n}{dy} \PY{o}{|} \PY{n}{x}\PY{o}{+}\PY{n}{y}
\PY{n}{platex}\PY{p}{(}\PY{n}{op}\PY{p}{)}
\end{Verbatim}
\end{tcolorbox}

    \begin{Verbatim}[commandchars=\\\{\}]
\textbackslash{}begin\{gather\}\textbackslash{}frac\{d\}\{d x\}\textbackslash{}hspace\{1mm\}\textbackslash{}frac\{d\^{}\{2\}\}\{d y\^{}2\}\textbackslash{}hspace\{1mm\}x +
y\textbackslash{}end\{gather\}
    \end{Verbatim}

    \[ \begin{gather}\frac{d}{d x}\hspace{1mm}\frac{d^{2}}{d y^2}\hspace{1mm}x + y\end{gather}
 \]

    You can use a more complex expression as differentiator too. Also shows
an example of not equal

    \begin{tcolorbox}[breakable, size=fbox, boxrule=1pt, pad at break*=1mm,colback=cellbackground, colframe=cellborder]
\prompt{In}{incolor}{13}{\boxspacing}
\begin{Verbatim}[commandchars=\\\{\}]
\PY{n}{x} \PY{o}{=} \PY{n}{makeVar}\PY{p}{(}\PY{l+s+s1}{\PYZsq{}}\PY{l+s+s1}{x}\PY{l+s+s1}{\PYZsq{}}\PY{p}{)}
\PY{n}{f} \PY{o}{=} \PY{p}{(}\PY{n+nb}{pow}\PY{p}{(}\PY{n}{x}\PY{p}{,}\PY{l+m+mi}{2}\PY{p}{)} \PY{o}{+} \PY{n}{x} \PY{o}{+} \PY{l+m+mi}{1}\PY{p}{)}
\PY{n}{df} \PY{o}{=} \PY{n}{makeDerivative}\PY{p}{(}\PY{p}{(}\PY{n}{f}\PY{p}{,}\PY{l+m+mi}{2}\PY{p}{)}\PY{p}{)}

\PY{n}{op} \PY{o}{=} \PY{n}{df} \PY{o}{|} \PY{n}{f} \PY{o}{!=} \PY{n}{Var}\PY{p}{(}\PY{l+s+s1}{\PYZsq{}}\PY{l+s+s1}{2}\PY{l+s+s1}{\PYZsq{}}\PY{p}{)}
\PY{n}{platex}\PY{p}{(}\PY{n}{op}\PY{p}{)}
\end{Verbatim}
\end{tcolorbox}

    \begin{Verbatim}[commandchars=\\\{\}]
\textbackslash{}begin\{gather\}\textbackslash{}frac\{d\^{}\{2\}\}\{d (x\^{}2 + x + 1)\^{}2\}\textbackslash{}hspace\{1mm\}x\^{}2 + x + 1 \textbackslash{}neq
2\textbackslash{}end\{gather\}
    \end{Verbatim}

    \[ \begin{gather}\frac{d^{2}}{d (x^2 + x + 1)^2}\hspace{1mm}x^2 + x + 1 \neq 2\end{gather} \]

    We used \texttt{\textbar{}} for space between the variables. Calling
\texttt{df} with \texttt{f} in the previous example will result in
parenthesizing the function on which the differentiation operator
acts.\\
\textbf{NOTE:} The \texttt{=} here is normal python assignment to a new
variable \texttt{op}. It wont be rendered as your latex. For that you
either use \texttt{@} operator or the \texttt{equals} method

    \begin{tcolorbox}[breakable, size=fbox, boxrule=1pt, pad at break*=1mm,colback=cellbackground, colframe=cellborder]
\prompt{In}{incolor}{14}{\boxspacing}
\begin{Verbatim}[commandchars=\\\{\}]
\PY{n}{op} \PY{o}{=} \PY{n}{df}\PY{p}{(}\PY{n}{f}\PY{p}{)} \PY{o}{!=} \PY{n}{makeVar}\PY{p}{(}\PY{l+s+s1}{\PYZsq{}}\PY{l+s+s1}{2}\PY{l+s+s1}{\PYZsq{}}\PY{p}{)}
\PY{n}{platex}\PY{p}{(}\PY{n}{op}\PY{p}{)}
\end{Verbatim}
\end{tcolorbox}

    \begin{Verbatim}[commandchars=\\\{\}]
\textbackslash{}begin\{gather\}\textbackslash{}frac\{d\^{}\{2\}\}\{d (x\^{}2 + x + 1)\^{}2\}(x\^{}2 + x + 1) \textbackslash{}neq 2\textbackslash{}end\{gather\}
    \end{Verbatim}

    \[ \begin{gather}\frac{d^{2}}{d (x^2 + x + 1)^2}(x^2 + x + 1) \neq 2\end{gather} \]

    Same example with Partial

    \begin{tcolorbox}[breakable, size=fbox, boxrule=1pt, pad at break*=1mm,colback=cellbackground, colframe=cellborder]
\prompt{In}{incolor}{15}{\boxspacing}
\begin{Verbatim}[commandchars=\\\{\}]
\PY{k+kn}{from} \PY{n+nn}{pytex} \PY{k+kn}{import} \PY{n}{makePartial}
\PY{n}{x} \PY{o}{=} \PY{n}{makeVar}\PY{p}{(}\PY{l+s+s1}{\PYZsq{}}\PY{l+s+s1}{x}\PY{l+s+s1}{\PYZsq{}}\PY{p}{)}
\PY{n}{f} \PY{o}{=} \PY{n+nb}{pow}\PY{p}{(}\PY{n}{x}\PY{p}{,}\PY{l+m+mi}{2}\PY{p}{)} \PY{o}{+} \PY{n}{x} \PY{o}{+} \PY{l+m+mi}{1}
\PY{n}{df} \PY{o}{=} \PY{n}{makePartial}\PY{p}{(}\PY{p}{(}\PY{n}{f}\PY{p}{,}\PY{l+m+mi}{3}\PY{p}{)}\PY{p}{)}

\PY{n}{op} \PY{o}{=} \PY{n}{df} \PY{o}{|} \PY{n}{f} \PY{o}{==} \PY{n}{Var}\PY{p}{(}\PY{l+s+s1}{\PYZsq{}}\PY{l+s+s1}{1}\PY{l+s+s1}{\PYZsq{}}\PY{p}{)}
\PY{n}{platex}\PY{p}{(}\PY{n}{op}\PY{p}{)}
\end{Verbatim}
\end{tcolorbox}

    \begin{Verbatim}[commandchars=\\\{\}]
\textbackslash{}begin\{gather\}\textbackslash{}frac\{\textbackslash{}partial \^{}\{3\}\}\{\textbackslash{}partial  (x\^{}2 + x + 1)\^{}3\}\textbackslash{}hspace\{1mm\}x\^{}2 + x
+ 1 = 1\textbackslash{}end\{gather\}
    \end{Verbatim}

    \[ \begin{gather}\frac{\partial ^{3}}{\partial  (x^2 + x + 1)^3}\hspace{1mm}x^2 + x + 1 = 1\end{gather} \]

    \hypertarget{matrices-and-vectors}{%
\subsection{Matrices and Vectors}\label{matrices-and-vectors}}

You can make both matrices and vectors with a single class
\texttt{Matrix} by giving appropriate dimensions\\
\texttt{MatrixBuilder} provides a builder pattern API to create a
Matrix.

For \texttt{Matrix}, pass a list representing a matrix with appropriate
dimensions

    \begin{tcolorbox}[breakable, size=fbox, boxrule=1pt, pad at break*=1mm,colback=cellbackground, colframe=cellborder]
\prompt{In}{incolor}{16}{\boxspacing}
\begin{Verbatim}[commandchars=\\\{\}]
\PY{k+kn}{from} \PY{n+nn}{pytex} \PY{k+kn}{import} \PY{n}{Matrix}
\PY{c+c1}{\PYZsh{} simple row vector}
\PY{n}{platex}\PY{p}{(}\PY{n}{Matrix}\PY{p}{(}\PY{p}{[}\PY{l+m+mi}{1}\PY{p}{,}\PY{l+m+mi}{2}\PY{p}{,}\PY{l+m+mi}{3}\PY{p}{]}\PY{p}{)}\PY{p}{)}
\end{Verbatim}
\end{tcolorbox}

    \begin{Verbatim}[commandchars=\\\{\}]
\textbackslash{}begin\{gather\}\textbackslash{}begin\{bmatrix\} 1 \&  2 \&  3 \textbackslash{}end\{bmatrix\}\textbackslash{}end\{gather\}
    \end{Verbatim}

    \[ \begin{gather}\begin{bmatrix} 1 &  2 &  3 \end{bmatrix}\end{gather} \]

    \begin{tcolorbox}[breakable, size=fbox, boxrule=1pt, pad at break*=1mm,colback=cellbackground, colframe=cellborder]
\prompt{In}{incolor}{17}{\boxspacing}
\begin{Verbatim}[commandchars=\\\{\}]
\PY{c+c1}{\PYZsh{} simple column vector}
\PY{n}{m} \PY{o}{=} \PY{p}{[}\PY{p}{[}\PY{l+m+mi}{1}\PY{p}{]}\PY{p}{,}\PY{p}{[}\PY{l+m+mi}{2}\PY{p}{]}\PY{p}{,}\PY{p}{[}\PY{l+m+mi}{3}\PY{p}{]}\PY{p}{]}
\PY{n}{platex}\PY{p}{(}\PY{n}{Matrix}\PY{p}{(}\PY{n}{m}\PY{p}{)}\PY{p}{)}
\end{Verbatim}
\end{tcolorbox}

    \begin{Verbatim}[commandchars=\\\{\}]
\textbackslash{}begin\{gather\}\textbackslash{}begin\{bmatrix\} 1 \textbackslash{}\textbackslash{} 2 \textbackslash{}\textbackslash{} 3 \textbackslash{}end\{bmatrix\}\textbackslash{}end\{gather\}
    \end{Verbatim}

    \[ \begin{gather}\begin{bmatrix} 1 \\ 2 \\ 3 \end{bmatrix}\end{gather}\]

    \begin{tcolorbox}[breakable, size=fbox, boxrule=1pt, pad at break*=1mm,colback=cellbackground, colframe=cellborder]
\prompt{In}{incolor}{18}{\boxspacing}
\begin{Verbatim}[commandchars=\\\{\}]
\PY{c+c1}{\PYZsh{} adding subscript and powers}
\PY{n}{platex}\PY{p}{(}\PY{n}{Matrix}\PY{p}{(}\PY{n}{m}\PY{p}{,} \PY{l+s+s1}{\PYZsq{}}\PY{l+s+s1}{3x1}\PY{l+s+s1}{\PYZsq{}}\PY{p}{,} \PY{l+m+mi}{2}\PY{p}{)}\PY{p}{)}
\end{Verbatim}
\end{tcolorbox}

    \begin{Verbatim}[commandchars=\\\{\}]
\textbackslash{}begin\{gather\}\textbackslash{}begin\{bmatrix\} 1 \textbackslash{}\textbackslash{} 2 \textbackslash{}\textbackslash{} 3 \textbackslash{}end\{bmatrix\}\^{}\{2\}\_\{3x1\}\textbackslash{}end\{gather\}
    \end{Verbatim}

    \[  \begin{gather}\begin{bmatrix} 1 \\ 2 \\ 3 \end{bmatrix}^{2}_{3x1}\end{gather} \]

    \begin{tcolorbox}[breakable, size=fbox, boxrule=1pt, pad at break*=1mm,colback=cellbackground, colframe=cellborder]
\prompt{In}{incolor}{19}{\boxspacing}
\begin{Verbatim}[commandchars=\\\{\}]
\PY{c+c1}{\PYZsh{} different types of brackets, default is [] \PYZhy{}\PYZgt{} square brackets}
\PY{c+c1}{\PYZsh{} Passing anything else in surround renders the matrix without a border}
\PY{n}{platex}\PY{p}{(}\PY{n}{Matrix}\PY{p}{(}\PY{n}{m}\PY{p}{,} \PY{n}{surround}\PY{o}{=}\PY{l+s+s1}{\PYZsq{}}\PY{l+s+s1}{()}\PY{l+s+s1}{\PYZsq{}}\PY{p}{)} \PY{o}{|} \PY{n}{Matrix}\PY{p}{(}\PY{n}{m}\PY{p}{,} \PY{n}{surround}\PY{o}{=}\PY{l+s+s1}{\PYZsq{}}\PY{l+s+s1}{||}\PY{l+s+s1}{\PYZsq{}}\PY{p}{)} \PY{o}{|} \PY{n}{Matrix}\PY{p}{(}\PY{n}{m}\PY{p}{,} \PY{n}{surround}\PY{o}{=}\PY{l+s+s1}{\PYZsq{}}\PY{l+s+s1}{||||}\PY{l+s+s1}{\PYZsq{}}\PY{p}{)}\PY{p}{)}
\end{Verbatim}
\end{tcolorbox}

    \begin{Verbatim}[commandchars=\\\{\}]
\textbackslash{}begin\{gather\}\textbackslash{}begin\{pmatrix\} 1 \textbackslash{}\textbackslash{} 2 \textbackslash{}\textbackslash{} 3
\textbackslash{}end\{pmatrix\}\textbackslash{}hspace\{1mm\}\textbackslash{}begin\{vmatrix\} 1 \textbackslash{}\textbackslash{} 2 \textbackslash{}\textbackslash{} 3
\textbackslash{}end\{vmatrix\}\textbackslash{}hspace\{1mm\}\textbackslash{}begin\{Vmatrix\} 1 \textbackslash{}\textbackslash{} 2 \textbackslash{}\textbackslash{} 3 \textbackslash{}end\{Vmatrix\}\textbackslash{}end\{gather\}
    \end{Verbatim}

    \[ \begin{gather}\begin{pmatrix} 1 \\ 2 \\ 3 \end{pmatrix}\hspace{1mm}\begin{vmatrix} 1 \\ 2 \\ 3 \end{vmatrix}\hspace{1mm}\begin{Vmatrix} 1 \\ 2 \\ 3 \end{Vmatrix}\end{gather} \]

    A more complex matrix example

    \begin{tcolorbox}[breakable, size=fbox, boxrule=1pt, pad at break*=1mm,colback=cellbackground, colframe=cellborder]
\prompt{In}{incolor}{20}{\boxspacing}
\begin{Verbatim}[commandchars=\\\{\}]
\PY{k+kn}{from} \PY{n+nn}{pytex} \PY{k+kn}{import} \PY{n}{makePartial}\PY{p}{,} \PY{n}{makeVar}\PY{p}{,} \PY{n}{Matrix}
\PY{k+kn}{from} \PY{n+nn}{pytex}\PY{n+nn}{.}\PY{n+nn}{platforms}\PY{n+nn}{.}\PY{n+nn}{jupyter} \PY{k+kn}{import} \PY{n}{latex}
\PY{n}{du}\PY{p}{,} \PY{n}{dv} \PY{o}{=} \PY{n}{makePartial}\PY{p}{(}\PY{l+s+s1}{\PYZsq{}}\PY{l+s+s1}{u}\PY{l+s+s1}{\PYZsq{}}\PY{p}{,} \PY{l+s+s1}{\PYZsq{}}\PY{l+s+s1}{v}\PY{l+s+s1}{\PYZsq{}}\PY{p}{)}
\PY{n}{X}\PY{p}{,} \PY{n}{Y} \PY{o}{=} \PY{n}{makeVar}\PY{p}{(}\PY{l+s+s1}{\PYZsq{}}\PY{l+s+s1}{X}\PY{l+s+s1}{\PYZsq{}}\PY{p}{,} \PY{l+s+s1}{\PYZsq{}}\PY{l+s+s1}{Y}\PY{l+s+s1}{\PYZsq{}}\PY{p}{)}
\PY{n}{l} \PY{o}{=} \PY{p}{[}
    \PY{p}{[}\PY{l+s+s1}{\PYZsq{}}\PY{l+s+s1}{i}\PY{l+s+s1}{\PYZsq{}}\PY{p}{,} \PY{l+s+s1}{\PYZsq{}}\PY{l+s+s1}{j}\PY{l+s+s1}{\PYZsq{}}\PY{p}{,} \PY{l+s+s1}{\PYZsq{}}\PY{l+s+s1}{k}\PY{l+s+s1}{\PYZsq{}}\PY{p}{]}\PY{p}{,}
    \PY{p}{[}\PY{n}{du}\PY{o}{|}\PY{n}{X}\PY{p}{,} \PY{n}{du}\PY{o}{|}\PY{n}{Y}\PY{p}{,} \PY{l+m+mi}{0}\PY{p}{]}\PY{p}{,}
    \PY{p}{[}\PY{n}{dv}\PY{o}{|}\PY{n}{X}\PY{p}{,} \PY{n}{dv}\PY{o}{|}\PY{n}{Y}\PY{p}{,} \PY{l+m+mi}{0}\PY{p}{]}
\PY{p}{]}
\PY{n}{platex}\PY{p}{(}\PY{n}{Matrix}\PY{p}{(}\PY{n}{l}\PY{p}{,} \PY{n}{subscript}\PY{o}{=}\PY{l+s+s1}{\PYZsq{}}\PY{l+s+s1}{3x3}\PY{l+s+s1}{\PYZsq{}}\PY{p}{,} \PY{n}{power}\PY{o}{=}\PY{l+m+mi}{2}\PY{p}{,}\PY{n}{surround}\PY{o}{=}\PY{l+s+s1}{\PYZsq{}}\PY{l+s+s1}{||}\PY{l+s+s1}{\PYZsq{}}\PY{p}{)}\PY{p}{)}
\end{Verbatim}
\end{tcolorbox}

    \begin{Verbatim}[commandchars=\\\{\}]
\textbackslash{}begin\{gather\}\textbackslash{}begin\{vmatrix\} i \&  j \&  k \textbackslash{}\textbackslash{} \textbackslash{}frac\{\textbackslash{}partial \}\{\textbackslash{}partial
u\}\textbackslash{}hspace\{1mm\}X \&  \textbackslash{}frac\{\textbackslash{}partial \}\{\textbackslash{}partial  u\}\textbackslash{}hspace\{1mm\}Y \&  0 \textbackslash{}\textbackslash{}
\textbackslash{}frac\{\textbackslash{}partial \}\{\textbackslash{}partial  v\}\textbackslash{}hspace\{1mm\}X \&  \textbackslash{}frac\{\textbackslash{}partial \}\{\textbackslash{}partial
v\}\textbackslash{}hspace\{1mm\}Y \&  0 \textbackslash{}end\{vmatrix\}\^{}\{2\}\_\{3x3\}\textbackslash{}end\{gather\}
    \end{Verbatim}

    \[ \begin{gather}\begin{vmatrix} i &  j &  k \\ \frac{\partial }{\partial  u}\hspace{1mm}X &  \frac{\partial }{\partial  u}\hspace{1mm}Y &  0 \\ \frac{\partial }{\partial  v}\hspace{1mm}X &  \frac{\partial }{\partial  v}\hspace{1mm}Y &  0 \end{vmatrix}^{2}_{3x3}\end{gather} \]

    Same example with \texttt{MatrixBuilder}

    \begin{tcolorbox}[breakable, size=fbox, boxrule=1pt, pad at break*=1mm,colback=cellbackground, colframe=cellborder]
\prompt{In}{incolor}{21}{\boxspacing}
\begin{Verbatim}[commandchars=\\\{\}]
\PY{c+c1}{\PYZsh{} from pytex import MatrixBuilder}
\PY{k+kn}{from} \PY{n+nn}{pytex} \PY{k+kn}{import} \PY{n}{makePartial}\PY{p}{,} \PY{n}{makeVar}\PY{p}{,} \PY{n}{Matrix}
\PY{k+kn}{from} \PY{n+nn}{pytex}\PY{n+nn}{.}\PY{n+nn}{platforms}\PY{n+nn}{.}\PY{n+nn}{jupyter} \PY{k+kn}{import} \PY{n}{latex}
\PY{n}{du}\PY{p}{,} \PY{n}{dv} \PY{o}{=} \PY{n}{makePartial}\PY{p}{(}\PY{l+s+s1}{\PYZsq{}}\PY{l+s+s1}{u}\PY{l+s+s1}{\PYZsq{}}\PY{p}{,} \PY{l+s+s1}{\PYZsq{}}\PY{l+s+s1}{v}\PY{l+s+s1}{\PYZsq{}}\PY{p}{)}
\PY{n}{X}\PY{p}{,} \PY{n}{Y} \PY{o}{=} \PY{n}{makeVar}\PY{p}{(}\PY{l+s+s1}{\PYZsq{}}\PY{l+s+s1}{X}\PY{l+s+s1}{\PYZsq{}}\PY{p}{,} \PY{l+s+s1}{\PYZsq{}}\PY{l+s+s1}{Y}\PY{l+s+s1}{\PYZsq{}}\PY{p}{)}

\PY{n}{m} \PY{o}{=} \PY{p}{(}\PY{n}{Matrix}\PY{o}{.}\PY{n}{builder}\PY{p}{(}\PY{p}{)}
    \PY{o}{.}\PY{n}{add}\PY{p}{(}\PY{l+s+s1}{\PYZsq{}}\PY{l+s+s1}{i}\PY{l+s+s1}{\PYZsq{}}\PY{p}{,}\PY{l+s+s1}{\PYZsq{}}\PY{l+s+s1}{j}\PY{l+s+s1}{\PYZsq{}}\PY{p}{,}\PY{l+s+s1}{\PYZsq{}}\PY{l+s+s1}{k}\PY{l+s+s1}{\PYZsq{}}\PY{p}{)}
    \PY{o}{.}\PY{n}{add}\PY{p}{(}\PY{n}{du}\PY{o}{|}\PY{n}{X}\PY{p}{,} \PY{n}{du}\PY{o}{|}\PY{n}{Y}\PY{p}{,} \PY{l+m+mi}{0}\PY{p}{)}
    \PY{o}{.}\PY{n}{add}\PY{p}{(}\PY{n}{dv}\PY{o}{|}\PY{n}{X}\PY{p}{,} \PY{n}{dv}\PY{o}{|}\PY{n}{Y}\PY{p}{,} \PY{l+m+mi}{0}\PY{p}{)}
    \PY{o}{.}\PY{n}{create}\PY{p}{(}\PY{n}{subscript}\PY{o}{=}\PY{l+s+s1}{\PYZsq{}}\PY{l+s+s1}{3x3}\PY{l+s+s1}{\PYZsq{}}\PY{p}{,} \PY{n}{power}\PY{o}{=}\PY{l+m+mi}{2}\PY{p}{,} \PY{n}{surround}\PY{o}{=}\PY{l+s+s1}{\PYZsq{}}\PY{l+s+s1}{||}\PY{l+s+s1}{\PYZsq{}}\PY{p}{)}\PY{p}{)}
\PY{n}{platex}\PY{p}{(}\PY{n}{m}\PY{p}{)}
\end{Verbatim}
\end{tcolorbox}

    \begin{Verbatim}[commandchars=\\\{\}]
\textbackslash{}begin\{gather\}\textbackslash{}begin\{vmatrix\} i \&  j \&  k \textbackslash{}\textbackslash{} \textbackslash{}frac\{\textbackslash{}partial \}\{\textbackslash{}partial
u\}\textbackslash{}hspace\{1mm\}X \&  \textbackslash{}frac\{\textbackslash{}partial \}\{\textbackslash{}partial  u\}\textbackslash{}hspace\{1mm\}Y \&  0 \textbackslash{}\textbackslash{}
\textbackslash{}frac\{\textbackslash{}partial \}\{\textbackslash{}partial  v\}\textbackslash{}hspace\{1mm\}X \&  \textbackslash{}frac\{\textbackslash{}partial \}\{\textbackslash{}partial
v\}\textbackslash{}hspace\{1mm\}Y \&  0 \textbackslash{}end\{vmatrix\}\^{}\{2\}\_\{3x3\}\textbackslash{}end\{gather\}
    \end{Verbatim}

    \[ \begin{gather}\begin{vmatrix} i &  j &  k \\ \frac{\partial }{\partial  u}\hspace{1mm}X &  \frac{\partial }{\partial  u}\hspace{1mm}Y &  0 \\ \frac{\partial }{\partial  v}\hspace{1mm}X &  \frac{\partial }{\partial  v}\hspace{1mm}Y &  0 \end{vmatrix}^{2}_{3x3}\end{gather} \]

    \texttt{MatrixWithDots} is also best explained by example

    \begin{tcolorbox}[breakable, size=fbox, boxrule=1pt, pad at break*=1mm,colback=cellbackground, colframe=cellborder]
\prompt{In}{incolor}{22}{\boxspacing}
\begin{Verbatim}[commandchars=\\\{\}]
\PY{k+kn}{from} \PY{n+nn}{pytex} \PY{k+kn}{import} \PY{n}{MatrixWithDots}\PY{p}{,} \PY{n}{makeVector}\PY{p}{,} \PY{n}{makeVar}\PY{p}{,} \PY{n}{makePartial}
\PY{c+c1}{\PYZsh{} Jacobian matrix}

\PY{n}{f}\PY{p}{,} \PY{n}{J} \PY{o}{=} \PY{n}{makeVector}\PY{p}{(}\PY{l+s+s1}{\PYZsq{}}\PY{l+s+s1}{f}\PY{l+s+s1}{\PYZsq{}}\PY{p}{,} \PY{l+s+s1}{\PYZsq{}}\PY{l+s+s1}{J}\PY{l+s+s1}{\PYZsq{}}\PY{p}{)}
\PY{n}{f1}\PY{p}{,} \PY{n}{fm} \PY{o}{=} \PY{n}{makeVar}\PY{p}{(}\PY{l+s+s1}{\PYZsq{}}\PY{l+s+s1}{f\PYZus{}1}\PY{l+s+s1}{\PYZsq{}}\PY{p}{,} \PY{l+s+s1}{\PYZsq{}}\PY{l+s+s1}{f\PYZus{}m}\PY{l+s+s1}{\PYZsq{}}\PY{p}{)}
\PY{n}{dx1}\PY{p}{,} \PY{n}{dxn} \PY{o}{=} \PY{n}{makePartial}\PY{p}{(}\PY{p}{(}\PY{l+s+s1}{\PYZsq{}}\PY{l+s+s1}{x\PYZus{}1}\PY{l+s+s1}{\PYZsq{}}\PY{p}{,}\PY{p}{)}\PY{p}{,} \PY{p}{(}\PY{l+s+s1}{\PYZsq{}}\PY{l+s+s1}{x\PYZus{}n}\PY{l+s+s1}{\PYZsq{}}\PY{p}{,}\PY{p}{)}\PY{p}{)}
\PY{n}{op} \PY{o}{=} \PY{n}{J} \PY{o}{==} \PY{n}{MatrixWithDots}\PY{p}{(}\PY{p}{[}\PY{p}{[}\PY{n}{dx1}\PY{o}{|}\PY{n}{f}\PY{p}{,} \PY{n}{dxn}\PY{o}{|}\PY{n}{f}\PY{p}{]}\PY{p}{]}\PY{p}{)}
\PY{n}{platex}\PY{p}{(}\PY{n}{op}\PY{p}{)}
\end{Verbatim}
\end{tcolorbox}

    \begin{Verbatim}[commandchars=\\\{\}]
\textbackslash{}begin\{gather\}\textbackslash{}vec\{\textbackslash{}mathbf\{J\}\} = \textbackslash{}begin\{bmatrix\} \textbackslash{}frac\{\textbackslash{}partial \}\{\textbackslash{}partial
x\_1\}\textbackslash{}hspace\{1mm\}\textbackslash{}vec\{\textbackslash{}mathbf\{f\}\} \&  \textbackslash{}cdots \&  \textbackslash{}frac\{\textbackslash{}partial \}\{\textbackslash{}partial
x\_n\}\textbackslash{}hspace\{1mm\}\textbackslash{}vec\{\textbackslash{}mathbf\{f\}\} \textbackslash{}end\{bmatrix\}\textbackslash{}end\{gather\}
    \end{Verbatim}

    \[ \begin{gather}\vec{\mathbf{J}} = \begin{bmatrix} \frac{\partial }{\partial  x_1}\hspace{1mm}\vec{\mathbf{f}} &  \cdots &  \frac{\partial }{\partial  x_n}\hspace{1mm}\vec{\mathbf{f}} \end{bmatrix}\end{gather} \]

    \begin{tcolorbox}[breakable, size=fbox, boxrule=1pt, pad at break*=1mm,colback=cellbackground, colframe=cellborder]
\prompt{In}{incolor}{23}{\boxspacing}
\begin{Verbatim}[commandchars=\\\{\}]
\PY{n}{ll} \PY{o}{=} \PY{p}{[}
    \PY{p}{[}\PY{n}{dx1} \PY{o}{|} \PY{n}{f1}\PY{p}{,} \PY{n}{dxn} \PY{o}{|} \PY{n}{f1}\PY{p}{]}\PY{p}{,}
    \PY{p}{[}\PY{n}{dx1} \PY{o}{|} \PY{n}{fm}\PY{p}{,} \PY{n}{dxn} \PY{o}{|} \PY{n}{fm}\PY{p}{]}\PY{p}{,}
\PY{p}{]}
\PY{n}{op2} \PY{o}{=} \PY{n}{op} \PY{o}{==} \PY{n}{MatrixWithDots}\PY{p}{(}\PY{n}{ll}\PY{p}{,} \PY{n}{shape}\PY{o}{=}\PY{p}{(}\PY{l+m+mi}{3}\PY{p}{,}\PY{l+m+mi}{3}\PY{p}{)}\PY{p}{)}
\PY{n}{platex}\PY{p}{(}\PY{n}{op2}\PY{p}{)}
\end{Verbatim}
\end{tcolorbox}

    \begin{Verbatim}[commandchars=\\\{\}]
\textbackslash{}begin\{gather\}\textbackslash{}begin\{bmatrix\} \textbackslash{}frac\{\textbackslash{}partial \}\{\textbackslash{}partial  x\_1\}\textbackslash{}hspace\{1mm\}f\_1 \&
\textbackslash{}cdots \&  \textbackslash{}frac\{\textbackslash{}partial \}\{\textbackslash{}partial  x\_n\}\textbackslash{}hspace\{1mm\}f\_1 \textbackslash{}\textbackslash{} \textbackslash{}vdots \&  \textbackslash{}ddots \&
\textbackslash{}vdots \textbackslash{}\textbackslash{} \textbackslash{}frac\{\textbackslash{}partial \}\{\textbackslash{}partial  x\_1\}\textbackslash{}hspace\{1mm\}f\_m \&  \textbackslash{}cdots \&
\textbackslash{}frac\{\textbackslash{}partial \}\{\textbackslash{}partial  x\_n\}\textbackslash{}hspace\{1mm\}f\_m \textbackslash{}end\{bmatrix\} = \textbackslash{}vec\{\textbackslash{}mathbf\{J\}\}
= \textbackslash{}begin\{bmatrix\} \textbackslash{}frac\{\textbackslash{}partial \}\{\textbackslash{}partial  x\_1\}\textbackslash{}hspace\{1mm\}\textbackslash{}vec\{\textbackslash{}mathbf\{f\}\} \&
\textbackslash{}cdots \&  \textbackslash{}frac\{\textbackslash{}partial \}\{\textbackslash{}partial  x\_n\}\textbackslash{}hspace\{1mm\}\textbackslash{}vec\{\textbackslash{}mathbf\{f\}\}
\textbackslash{}end\{bmatrix\}\textbackslash{}end\{gather\}
    \end{Verbatim}

    \[ \begin{gather}\begin{bmatrix} \frac{\partial }{\partial  x_1}\hspace{1mm}f_1 &  \cdots &  \frac{\partial }{\partial  x_n}\hspace{1mm}f_1 \\ \vdots &  \ddots &  \vdots \\ \frac{\partial }{\partial  x_1}\hspace{1mm}f_m &  \cdots &  \frac{\partial }{\partial  x_n}\hspace{1mm}f_m \end{bmatrix} = \vec{\mathbf{J}} = \begin{bmatrix} \frac{\partial }{\partial  x_1}\hspace{1mm}\vec{\mathbf{f}} &  \cdots &  \frac{\partial }{\partial  x_n}\hspace{1mm}\vec{\mathbf{f}} \end{bmatrix}\end{gather} \]

    \texttt{MatrixWithDots} is good with placing dots only when the list
size is not the same as the shape passed. By default, it makes one extra
row and one extra column unless size of the dimension is 1, in that case
it makes a single row/column. For more than 1 size, it puts the last
element in the row to the last column at that row and puts dots between
them. Same for columns

    The library provides many of the LaTeX math symbols by the same name
with some exceptions in case of a name clash with python keyword (eg.
\texttt{lambda})

    \begin{tcolorbox}[breakable, size=fbox, boxrule=1pt, pad at break*=1mm,colback=cellbackground, colframe=cellborder]
\prompt{In}{incolor}{25}{\boxspacing}
\begin{Verbatim}[commandchars=\\\{\}]
\PY{k+kn}{from} \PY{n+nn}{pytex} \PY{k+kn}{import} \PY{n}{Greek} \PY{k}{as} \PY{n}{G}\PY{p}{,} \PY{n}{Set} \PY{k}{as} \PY{n}{st}\PY{p}{,} \PY{n}{Operator} \PY{k}{as} \PY{n}{opr}
\end{Verbatim}
\end{tcolorbox}

    \begin{tcolorbox}[breakable, size=fbox, boxrule=1pt, pad at break*=1mm,colback=cellbackground, colframe=cellborder]
\prompt{In}{incolor}{26}{\boxspacing}
\begin{Verbatim}[commandchars=\\\{\}]
\PY{n}{op} \PY{o}{=} \PY{n}{G}\PY{o}{.}\PY{n}{beta} \PY{o}{|} \PY{n}{st}\PY{o}{.}\PY{n}{subseteq} \PY{o}{|} \PY{n}{G}\PY{o}{.}\PY{n}{Delta} \PY{o}{|} \PY{n}{opr}\PY{o}{.}\PY{n}{approxeq} \PY{o}{|} \PY{n}{Var}\PY{p}{(}\PY{l+s+s1}{\PYZsq{}}\PY{l+s+s1}{x}\PY{l+s+s1}{\PYZsq{}}\PY{p}{)} \PY{o}{+} \PY{l+m+mi}{1}
\PY{n}{platex}\PY{p}{(}\PY{n}{op}\PY{p}{)}
\end{Verbatim}
\end{tcolorbox}

    \begin{Verbatim}[commandchars=\\\{\}]
\textbackslash{}begin\{gather\}\textbackslash{}beta\textbackslash{}hspace\{1mm\}\textbackslash{}subseteq\textbackslash{}hspace\{1mm\}\textbackslash{}Delta\textbackslash{}hspace\{1mm\}\textbackslash{}approxeq\textbackslash{}
hspace\{1mm\}x + 1\textbackslash{}end\{gather\}
    \end{Verbatim}

    \[ \begin{gather}\beta\hspace{1mm}\subseteq\hspace{1mm}\Delta\hspace{1mm}\approxeq\hspace{1mm}x + 1\end{gather} \]


    % Add a bibliography block to the postdoc
    
    
    
\end{document}
